% Indicate the main file. Must go at the beginning of the file.
% !TEX root = ../main.tex

%%%%%%%%%%%%%%%%%%%%%%%%%%%%%%%%%%%%%%%%%%%%%%%%%%%%%%%%%%%%%%%%%%%%%%%%%%%%%%%%
% Abstract
%%%%%%%%%%%%%%%%%%%%%%%%%%%%%%%%%%%%%%%%%%%%%%%%%%%%%%%%%%%%%%%%%%%%%%%%%%%%%%%%

\vspace*{\fill}

\section*{Abstract}
\label{abstract}

Understanding the distribution of impervious and pervious surfaces is critical 
for effective urban planning, environmental management, and rainfall impact analysis. 
This study explores the use of convolutional neural networks (CNNs) for 
pixel-based classification of aerial remote sensing data to assess surface sealing. 
Leveraging high-resolution SwissImage RS data, the analysis employs a simplified 
ResNet-18 architecture adapted for four-channel inputs, including RGB and 
near-infrared bands. A comprehensive workflow was developed, encompassing 
data preprocessing, augmentation, and hyperparameter tuning. The best-performing 
model achieved a classification accuracy of 0.927 for simplified surface perviousness, 
demonstrating the potential of deep learning to improve upon traditional 
geoprocessing methods. While challenges such as mixed pixels and class imbalances remain, 
this research highlights promising avenues for future advancements 
in remote sensing through the integration of advanced neural architectures and self-supervised learning.

\vspace*{\fill}