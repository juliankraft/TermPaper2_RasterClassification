% Indicate the main file. Must go at the beginning of the file.
% !TEX root = ../main.tex

%%%%%%%%%%%%%%%%%%%%%%%%%%%%%%%%%%%%%%%%%%%%%%%%%%%%%%%%%%%%%%%%%%%%%%%%%%%%%%%%
% Abstract
%%%%%%%%%%%%%%%%%%%%%%%%%%%%%%%%%%%%%%%%%%%%%%%%%%%%%%%%%%%%%%%%%%%%%%%%%%%%%%%%

\vspace*{\fill}

\section*{Abstract}
\label{abstract}

Understanding the distribution of impervious and pervious surfaces is critical 
for effective urban planning, environmental management, and rainfall impact analysis. 
This study explores the use of convolutional neural networks (CNNs) for 
pixel-based classification of aerial remote sensing data to assess surface sealing. 
Using high-resolution SwissImage RS data, the analysis employs a simplified 
ResNet-18 architecture adapted for four-channel inputs, including RGB and 
near-infrared bands. A detailed workflow was developed, describing the 
preprocessing of the data, the training, and the evaluation of the model. 
A data augmentation strategy was implemented to improve the model's performance, 
and a hyperparameter tuning process was conducted to optimize the model. 
The best-performing model achieved a classification accuracy of 0.927, 
which is in a similar range to the results of previous studies 
utilizing a geoprocessing approach. While challenges such as mixed-pixel 
problems or limited data availability remain, this study demonstrates 
the potential of deep learning for detailed surface sealing analysis.

\vspace*{\fill}
