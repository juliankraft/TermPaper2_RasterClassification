% Indicate the main file. Must go at the beginning of the file.
% !TEX root = ../main.tex

%%%%%%%%%%%%%%%%%%%%%%%%%%%%%%%%%%%%%%%%%%%%%%%%%%%%%%%%%%%%%%%%%%%%%%%%%%%%%%%%
% 01-introduction
%%%%%%%%%%%%%%%%%%%%%%%%%%%%%%%%%%%%%%%%%%%%%%%%%%%%%%%%%%%%%%%%%%%%%%%%%%%%%%%%

\section{Introduction}
\label{introduction}

Understanding surface sealing is a critical aspect of urban planning, environmental 
monitoring, and sustainable development. Sealed surfaces, such as roads, parking lots, 
and buildings, reduce natural soil permeability, disrupt water infiltration, and 
contribute to urban heat island effects, flooding, and habitat loss. Accurate detection 
and mapping of sealed surfaces at high spatial resolutions are essential for informed 
decision-making and policy development.

Traditionally, remote sensing techniques have been employed to address this challenge. 
These methods rely on spectral data, including near-infrared (NIR) bands, which are 
particularly effective for differentiating between sealed and unsealed surfaces due to 
their ability to capture subtle variations in surface reflectance. However, conventional 
approaches often involve rule-based geoprocessing or manual interpretation, which can be 
time-consuming and limited in scalability \autocite{kadhimAdvancesRemoteSensing2016}.

In recent years, the advent of deep learning has revolutionized remote sensing by 
enabling more automated, accurate, and scalable analysis. Neural networks have proven 
particularly effective in tasks like image classification, semantic segmentation, and 
object detection. While many studies in this domain focus on object-based or scene-based 
approaches \autocite{thapaDeepLearningRemote2023}, pixel-level classification is 
an area with significant potential for granular analysis of surface characteristics. By 
leveraging the contextual information of neighboring pixels, pixel-based methods can 
provide a more detailed understanding of surface sealing, which is vital for 
applications requiring high spatial accuracy \autocite{zhengHighSpatialResolution2023}.

Convolutional Neural Networks (CNN) have been widely utilized for remote sensing applications, 
including surface sealing detection and land cover classification. 
Prominent examples include ResNet-18 and ResNet-50, which have demonstrated strong performance 
in tasks requiring high-resolution image classification due to their ability to effectively 
learn hierarchical feature representations \autocite{natyaDeepTransferLearning2022}. 
VGG19, known for its simplicity and depth, has also been employed in remote sensing studies
to classify urban landscapes and detect sealed surfaces, benefiting from its consistent 
feature extraction capabilities \autocite{alemTransferLearningModels2022}. 
These architectures highlight the effectiveness of deep CNNs in 
capturing both spatial and contextual information for detailed surface analysis.




\subsection{Background and Methodology}

This term paper builds on previous research conducted for the canton of Basel-Landschaft, 
Switzerland. The original study employed a geoprocessing approach to determine the 
perviousness of surfaces on a per-pixel basis using aerial imagery, including NIR bands. 
This method focused on leveraging spectral characteristics to classify surfaces, 
achieving valuable insights into land cover and sealing patterns.

This paper explores an alternative methodology by applying a CNN, a simplified ResNet-18
architecture, to the same dataset. CNNs have 
demonstrated remarkable success in image analysis tasks by learning hierarchical feature 
representations directly from raw data. By training a neural network on the aerial 
imagery, this study aims to automate the process of surface classification to determine
perviousness.

Unlike modern deep learning approaches that emphasize object-based or scene-based 
classifications \autocite{thapaDeepLearningRemote2023}, this research adopts a 
pixel-based classification framework. Pixel-based classification not only allows for 
finer spatial resolution but also incorporates contextual information from neighboring 
pixels. This is achieved by inputting small patches of image data into the network, 
enabling it to capture local spatial patterns and textures critical for distinguishing 
between sealed and unsealed surfaces.

This study contributes to the growing body of research on deep learning in remote 
sensing by demonstrating the feasibility and potential of a simplified ResNet-18 model 
for pixel-level classification. Moreover, the inclusion of NIR data ensures that the 
model can effectively differentiate between surface types based on their spectral 
properties, a key advantage for urban and environmental applications. The results of 
this study aim to provide a foundation for future work in this area, such as integrating 
multi-temporal data or experimenting with more advanced architectures to further enhance 
performance.