% Indicate the main file. Must go at the beginning of the file.
% !TEX root = ../main.tex

%%%%%%%%%%%%%%%%%%%%%%%%%%%%%%%%%%%%%%%%%%%%%%%%%%%%%%%%%%%%%%%%%%%%%%%%%%%%%%%%
% 01-introduction
%%%%%%%%%%%%%%%%%%%%%%%%%%%%%%%%%%%%%%%%%%%%%%%%%%%%%%%%%%%%%%%%%%%%%%%%%%%%%%%%

\section{Introduction}
\label{introduction}

This term paper is following up on some research done for the canton of Basel-Landschaft, Switzerland.
The goal of the original research was to categorize landcover data into pervious and impervious areas.
Originally a geo data processing approach was used to achieve this goal.
% Include some results from the original research

In this paper a different approach is described - a neural network is used to classify the landcover data.
This is a promising and rapidly evolving technique in the field of remote sensing.
While modern approaches focus on a Object or Scene based classification \autocite{thapaDeepLearningRemote2023},
in this case a pixel based classification is performed -- factoring in the information of a range of neighboring pixels.

