% Indicate the main file. Must go at the beginning of the file.
% !TEX root = ../main.tex

%%%%%%%%%%%%%%%%%%%%%%%%%%%%%%%%%%%%%%%%%%%%%%%%%%%%%%%%%%%%%%%%%%%%%%%%%%%%%%%%
% 01-introduction
%%%%%%%%%%%%%%%%%%%%%%%%%%%%%%%%%%%%%%%%%%%%%%%%%%%%%%%%%%%%%%%%%%%%%%%%%%%%%%%%

\section{Introduction}
\label{introduction}

Understanding surface sealing is a critical aspect of urban planning, environmental 
monitoring, and sustainable development. Sealed surfaces, such as roads, parking lots, 
and buildings, are impervious to water and can change the hydrological properties of
an area. Water cannot infiltrate the ground, reducing groundwater recharge and
increasing the risk of flooding. In addition, the collected water flows together with
wastewater into water treatment plants, which can lead to overloading. This
can result in untreated water being released into the environment. Therefore,
knowledge about the perviousness of surfaces is essential for informed decision-making
and policy development.

Traditionally, remote sensing techniques have been used to address this challenge. 
These methods rely on spectral data, including near-infrared (NIR) bands, which are 
particularly effective for classifying sealed and unsealed surfaces. 
However, conventional approaches often involve rule-based geoprocessing or manual interpretation, 
which can be time-consuming and limited in scalability \autocite{kadhimAdvancesRemoteSensing2016}.

In recent years, advancements in deep learning have revolutionized remote sensing by 
enabling more automated, accurate, and scalable analysis. This is heavily driven
by rapid improvements in computational power and the availability of large
datasets. Neural networks have demonstrated great potential for tasks like image classification,
segmentation, and object detection. While many studies in this domain focus on object-based or scene-based 
approaches \autocite{thapaDeepLearningRemote2023}, pixel-level classification is 
an area with significant potential for granular analysis of surface characteristics.
By factoring in the information of neighboring pixels, pixel-based methods can 
potentially provide very detailed analyses of an area, especially if high spatial
resolution spectral data is available \autocite{zhengHighSpatialResolution2023}.

Convolutional Neural Networks (CNNs) have been widely used for remote sensing applications, 
including surface sealing detection and land cover classification. 
Prominent examples include the use of models like ResNet-18 and ResNet-50, which have demonstrated strong performance 
in tasks requiring high-resolution image classification due to their ability to effectively 
learn hierarchical feature representations \autocite{natyaDeepTransferLearning2022}. 
VGG19, known for its simplicity and depth, has also been employed in remote sensing studies
to classify urban landscapes and detect sealed surfaces, benefiting from its consistent 
feature extraction capabilities \autocite{alemTransferLearningModels2022}. 
These architectures highlight the effectiveness of deep CNNs in 
capturing both spatial and contextual information for detailed surface analysis.

\subsection{Background and Methodology}

This term paper builds on previous research conducted by researchers at the
Zurich University of Applied Sciences (ZHAW) for the canton of Basel-Landschaft, 
Switzerland. The original study examines a geoprocessing approach to determine the 
perviousness of surfaces on a pixel level. The study combined high-resolution SwissImage
remote sensing (RS) data with survey data identifying building footprints, roads, and other sealed surfaces.
This paper explores an alternative methodology by feeding the RS data into a CNN
to classify surface sealing. CNNs have demonstrated great potential for image
classification by learning hierarchical feature representations directly from raw data \autocite{zhaoReviewConvolutionalNeural2024}.
Successfully training a CNN to detect sealed surfaces would provide a cost-effective
and scalable alternative to the geoprocessing approach.

Unlike modern deep learning approaches that often focus on object-based or scene-based 
classifications \autocite{thapaDeepLearningRemote2023}, this research examines a 
pixel-based classification approach. Pixel-based classification allows for a
very fine-grained analysis. This is especially useful when high-resolution data is
available, as with the SwissImage data with a resolution of 10 cm. By inputting small
patches of pixels, the information of neighboring pixels can be used to improve the
classification.

This study employs a simplified ResNet-18 architecture, adapted to accept four-channel
inputs, including RGB and NIR bands. The labeled data from the original research
is used both to train and test the model. The results will provide an indication
of the potential of deep learning for surface sealing classification, and in
the best case, a viable alternative to the geoprocessing approach.
