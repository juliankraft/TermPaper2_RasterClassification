% Indicate the main file. Must go at the beginning of the file.
% !TEX root = ../main.tex

%%%%%%%%%%%%%%%%%%%%%%%%%%%%%%%%%%%%%%%%%%%%%%%%%%%%%%%%%%%%%%%%%%%%%%%%%%%%%%%%
% 01-introduction
%%%%%%%%%%%%%%%%%%%%%%%%%%%%%%%%%%%%%%%%%%%%%%%%%%%%%%%%%%%%%%%%%%%%%%%%%%%%%%%%

\section{Introduction}
\label{introduction}

This term paper is following up on some research done for the canton of Basel-Landschaft, Switzerland.
In the original research, a geo processing approach was tested to determine the perviousness
per pixel using aerial imagery including near-infrared (NIR).


In this paper an alterative approach is tested - using the same data set, 
a neural network is trained to classify the aerial imagery.
This is a promising and rapidly evolving technique in the field of remote sensing.
While modern approaches focus on a Object or Scene based classification \autocite{thapaDeepLearningRemote2023},
in this case a pixel or patch based classification is performed -- factoring in the information of a range of neighboring pixels.

